\subsection{Xavier P. Burgos-Artizzu - Robust Cascaded Pose Regression}
RCPR metoda za detekciju karakterističnih točaka lica je poboljšanje metode CPR (\textit{Cascaded pose regression}). CPR metoda je vrlo efikasna i precizna u određivanju karakterističnih točaka lica, ali preciznost joj drastično opada kada se određuju karakteristične točke na licima koje su prekrivene s nekom preprekom. 

CPR uči kaskadu regresora R\textsuperscript{1\ldots T} koji progresivno transformiraju inicijalni skup karakterističnih točaka S\textsuperscript{0} u finalni odabira karakterističnih točaka S\textsuperscript{T}. Skup točaka S\textsuperscript{i} je definiran kao $ S_p^i = [x_p, y_p], p \in 1\ldots P$. Svaki od regresora $R^t$ producira pomak osnovne konfiguracije točaka $ \delta S^t$ koji se kombinira s ulazom tog regresora i stvara se ulaz u slijedeći regresor $S^t = S^{t-1} + \delta S^t$. 

RCPR metoda zahtijeva da označene karakteristične točke imaju informaciju o tome je li karakteristična točka prekrivena nekom zaprekom (naočale, ruka ...) ili ne. Time je definirana karakteristična točka kao $S_p^i = [x_p, y_p, v_p]$, gdje je $v_p$ realna vrijednost iz intervala $[0, 1]$.

Slika se podijeli u $3 \times 3$ polje. Svako polje ima informaciju o postotku prepreka unutar polja dobivenu kao procjena temeljena na $S^{t-1}$ karakterističnih točaka. U svakom koraku $t$ se uči $S_tot$ regresora koji se treniraju samo na 1 od 9 predefiniranih polja (svakom regresoru $R_i^t$ je nasumično dodijeljeno polje). Regresori generiraju pomake $\delta S_{1\ldots tot}$ iz kojih se računa težinski prosjek, gdje su težine obrnuto proporcionalne količini zapreka u polju na kojem je treniran regresor(za dobre rezultate dovoljno je koristiti $S_tot = 3$ regresora). Na kraju kaskade se dobiva finalni skup točaka $S_p^T = [x_p, y_p, v_p]$ te je potrebno odrediti prag $\tau$ koji označava je li točka $S_p$ prekrivena ili ne.

Metoda je testirana na tri različita skupa podataka (\textbf{LFPW}, \textbf{HELEN}, \textbf{LFW}) u kojima lica nisu prekrivena i postigla je bolje rezultate u odnosu na prijašnje metode, te je testirana na skupu podataka \textbf{COFW} u kojem su karakteristične točke lica djelomično prekrivene te je postigla zadovoljavajuće  rezultate.