\documentclass[times, utf8, diplomski]{fer}
\usepackage{booktabs}
\usepackage{indentfirst}
\usepackage{listings}
\usepackage{amsfonts}
\usepackage{amssymb}
\usepackage{comment}
\usepackage{graphicx}
\usepackage{mdframed}
\usepackage{pdfpages}
\usepackage{amsmath}
\usepackage{array}
\usepackage{underscore}
\usepackage[section]{placeins}
\usepackage{relsize}
\usepackage{hyperref}
\usepackage{amsmath}

%Engleski termini
\newcommand{\eng}[1]{(eng. \textit{#1})}

\begin{document}

\title{ Lokalizacija karakterističnih točaka lica u videu }

\author{ \begin{tabular}{ l }
	Generalić Boris \\
	Gulan Filip \\
	Kopljar Damir \\
	Miličević Andrija \\
	Nuić Hrvoje \\
	Šarić Fredi \\
	Zadro Tvrtko \\
\end{tabular}  }

\maketitle
\tableofcontents

\chapter{Projektni zadatak}

\section{Opis projektnog zadatka}

Lokalizacija karakterističnih točaka lica u videu ili fotografiji je tehnika koja se danas koristi u mnogim sustavima i uređajima. Susrećemo je na raznim društvenim servisima, poput \textit{Facebook-a}, koji ju koriste za automatsko označavanje ljudi na fotografijama. Većina algoritama lokalizacije točaka lica su iznimno kompleksni i zahtjevaju veliku količinu procesorske snage i memorije, pa je težnja usmjerena na poboljšavanje tih algoritama. No razvojem i napretkom tehnologije algoritmi lokalizacije točaka lica se danas uspješno, bez velikih problema, izvode i na mobilnim uređajima koji ih koriste u raznoraznim aplikacijama poput alata za šminkanje gdje osoba može uz pomoć praćenja lica vidjeti kako bi izgledali s određenim bojama na svom licu.

Kroz ovaj projekt pokušat će se dani problem lokalizacije karakterističnih točaka lica riješiti uporabom dubokih neuronskih mreža.

\section{Pregled i opis srodnih rješenja}

Iscrpan pregled srodne literature s predloženim rješenjima. Opis postojećih ispitnih baza (linkovi na javno dostupne baze).

\section{Konceptualno rješenje zadatka}

Sam sustav za lokalizaciju karakterističnih točaka lica je podijeljen u više segmenata, tj. podsustava. Prvi segment sustava na ulaz prima sliku ili jedan vremenski okvir video isječka. Dana slika ili isječak se zatim pretvaraju u sliku sivih nijansi. Tako obrađena slika se dovodi na ulaz podsustava za izlučivanje položaja svih lica na slici te kao rezultat vraća listu u obliku: koordinate gornjeg lijevog ugla, širina i visina lica.

Tako dobivena lista se zatim iskoristi na način da se iz slike sivih nijansi izrežu prepoznata lica i skaliraju. Pojedina skalirana lica dovede se na ulaze duboke neuronske mreže koja kao izlaze daje koordinate odabranih karakterističnih točaka lica. Tako dobivene točke skaliraju se u prostor početne slike ili isječka te se iscrtavaju i prikazuju korisniku sustava. 

\chapter{Postupak rješavanja zadatka}

(do 10 stranica)

Navesti numerirani slijed koraka rješavanja. Npr.: 1. Dobivanje binarne slike iz slike u boji, 2. Segmentacija objekata na slici, 3. Nalaženje rubova u slici ...

\section{Prvi korak}

Za svaki korak napisati što su ulazi i što su izlazi. Popisati sve algoritme/ koncepte koji se u tom koraku koriste za pretvorbu ulaza u izlaz. Navesti sve probleme koji su se pojavili u pojedinom koraku i kako su riješeni. Pojedinačno opisati svaki korišteni algoritam/koncept:

\subsection{Prvi algoritam}
Opis/koraci/matematička formulacija, prednosti i mane, ulazi i izlazi te korišteni parametri.
\subsection{Drugi algoritam}
Opis/koraci/matematička formulacija, prednosti i mane, ulazi i izlazi te korišteni parametri.

\section{Drugi korak}
...

\chapter{Ispitivanje rješenja}
(do 10 stranica)

\section{Ispitna baza}
Opisati ispitnu bazu, tipove i broj različitih uzoraka u bazi te na koji su način uzorci iz baze korišteni prilikom učenja i ispitivanja rješenja projektnog zadatka. 

\section{Rezultati učenja i ispitivanja}
Prikazati statističke podatke o uspješnosti rješenja prilikom učenja/ispitivanja te opisati eksperimente na temelju kojih su podaci dobiveni.

\section{Analiza rezultata}

Analizirati uzroke rezultata ispitivanja, povezati sa uzorcima u bazi i algoritmima korištenim u rješenju. Raspraviti moguća poboljšanja.

\chapter{Opis programske implementacije rješenja}

Opisati sučelje programske implementacije i način korištenja implementacije.


\chapter{Zaključak}

(do 2 stranice)

Ocijeniti uspješnost implementacije, navesti budući rad u smislu potrebnih poboljšanja. 


\chapter{Literatura}

1. Ime i prezime autora: Naziv časopisa vol. br. godina izdanja, pp od-do (npr. pp 486-492)/knjige/članka/web resursa (s linkom i datumom pristupa web resursu)
...
.
.
DVD/CD  
.
kompletan tekst projekta
izvorni kod projekta
exe verzija
readme file – upute za korištenje i pokretanje programa
.
baze slika (sve koje su korištene)
E-oblik članaka koji su korišteni za izradu projekta
primjeri obrade
..


%%\bibliography{literatura}
%%\bibliographystyle{fer}

\end{document}
